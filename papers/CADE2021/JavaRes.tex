\documentclass{llncs}

\usepackage{booktabs}
\usepackage{amssymb}
\usepackage{amsfonts}
\usepackage{graphicx}
\usepackage{url}
\usepackage{listings}
\usepackage{hyperref}
\usepackage{wrapfig}

\lstset{language=Python,showstringspaces=false}

%\usepackage{caption}
%\captionsetup[table]{skip=10pt}
%\setlength{\belowcaptionskip}{-10pt}

%\title{Teaching Automated Theorem Proving by Example: PyRes Pedagogical Prover\\(system description)}
\title{JavaRes}

\author{Adam Pease\inst{2}
        \and Stephan Schulz\inst{1}
  }
\institute{
  Articulate Software, USA,
  \email{\tt apease@articulatesoftware.com}
  \and
  DHBW Stuttgart, Germany,
  \email{\tt schulz@eprover.org}
}


\titlerunning{JavaRes}
%\authorrunning{Schulz, Cruanes and Vukmirovic}

\renewcommand{\textfraction}{.01}
\renewcommand{\topfraction}{.99}

\newcommand{\mw}[1]{\ensuremath{\mathit{#1}}}
\newcommand{\nat}{\ensuremath{\mathbb N}}
\newcommand{\integer}{\ensuremath{\mathbb Z}}
\newcommand{\rat}{\ensuremath{\mathbb Q}}
\newcommand{\real}{\ensuremath{\mathbb R}}
\newcommand{\eqn}[2]{\ensuremath{#1\!\simeq\!#2}}
\newcommand{\neqn}[2]{\ensuremath{#1\!\not\simeq\!#2}}
\newcommand{\ueqn}[2]{\ensuremath{#1\dot{\simeq}#2}}

\newcommand{\limpl}{\rightarrow}
\newcommand{\limplies}{\rightarrow} % I mistype this too often ;-)
\newcommand{\ltrue}{\ensuremath{\top}}
\newcommand{\lfalse}{\ensuremath{\bot}}
\newcommand{\lequiv}{\ensuremath{\leftrightarrow}}

\newcommand{\terms}{\ensuremath{\mathit{Term(F,V)}}}
\newcommand{\tops}{\mathop{\mathrm{top}}\nolimits}
\newcommand{\pos}{\mathop{\mathrm{pos}}\nolimits}
\newcommand{\gfpf}{\mathop{\mathrm{gfpf}}\nolimits}
\newcommand{\fpf}{\mathop{\mathrm{fpf}}\nolimits}
\newcommand{\fp}{\mathop{\mathrm{fp}}\nolimits}
\newcommand{\text}[1] {\mbox{#1}}

\renewcommand{\textfraction}{.05}
\renewcommand{\topfraction}{.95}
\renewcommand{\bottomfraction}{.95}

\newcommand{\GInferenzC}[3]
{
\begin{tabular}{c}
  $#1$ \\
  \hline
  \raisebox{-0.4ex}{$#2$} \\
\end{tabular}
  #3
}

\newcommand{\CInferenz}[2]
{
\begin{tabular}{c}
  $#1$ \\
  \hline
  \hline
  \raisebox{-0.4ex}{$#2$} \\
\end{tabular}
}

%\pagestyle{empty}
%\bibliographystyle{alpha}
\bibliographystyle{splncs04}

\begin{document}

\maketitle


JavaRes is a demonstration prover patterned after
PyRes~\cite{SP:IJCAR-2020}.  In this paper we discuss in more detail
the architecture and data structures of this prover and the experience
of one of us (Pease) implementing the prover, without prior expertise
in writing an FOL prover.  We provide performance metrics relative to
PyRes and other well-known provers.  To illustrate the value of
JavaRes for learning about theorem proving we also describe
implementation of an alternate simple clausification algorithm,
graphical proof presentation, implementation of the SInE axiom
selection algoithm, and use with the SUMO theory and syntax.

Automated theorem proving is a fascinating and useful discipline but
can be mystifying for someone not deeply acquainted with the field.
It should be fair to say that most computer science professionals do
not understand the power of inference in FOL and how it provides
distinct capabilities from simpler representation such as graphs or
description logics.  Part of the reason for this lack of general
familiarity may be because the barier to entry in the field remains
high, despite many decades of work and publication.  Most publications
require a degree of mathematical sophistication to understand, and
even with such capability, a reader will not know which data
structures to use, or which of the many algorithms with be simplest to
implement.



We intially began with just an attempt for one of us (Pease) to learn
about ATP, motivated by decades of work in formal ontology (Pease,
2011), finding that among many excellent books, including (Harrison)
the first steps to understand ATP were too high.  Fortunately, the
creator of the Eprover (Schulz), provides the explanations needed to
understand how and where to begin, and this grew into creation of
PyRes.  To see whether PyRes provided enough structure, we wrote
JavaRes and then used it as a platform to implement several extensions
and see whether both provers could provide a suitable platform for ATP
education.


\bibliography{stsbib}

\end{document}

%%% Local Variables:
%%% mode: latex
%%% eval: (tex-pdf-mode)
%%% TeX-master: t
%%% End:
